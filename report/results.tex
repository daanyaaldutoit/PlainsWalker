\section{Discussion of the Results}

\subsection{Areas for Improvement}

The \emph{PlainsWalker} simulation meets most of its goals -- visibly, much of the required functionality is, at least in part, available. However, there is a lot that requires improvement and many ways that we did not achieve our goals.

\subsubsection{Efficiency}
Our simulation slows down drastically for large numbers of herd members, which makes it frustrating to use and in some cases may cause it to crash as it exceeds the memory space assigned to the JVM. Despite our efforts to streamline the programme, it is still very slow and could probably be severely improved by some clever techniques that we didn't think of or manage to research in time.

\subsubsection{Functionality}
There are several core functions that we didn't manage to include. These largely revolve around the editing of placed objects, such as moving them between assigned groups, deleting them and changing their starting positions.

\subsubsection{Usability}
The usability is the section we most feel we could have improved on, having had several ideas that we didn't manage to finish in time.

\noindent Specifically, we wanted to improve the user interface and add a significant amount of visual feedback to recognise the user's action. We had several ideas, including changing the cursor image and window background colour depending on the insertion mode (e.g. herd insertion vs. predator insertion). 

\subsubsection{Linking}
Although python scripts and an export script were written, we would have preferred to have created a nicer and more user-friendly link between \emph{PlainsWalker} and \emph{SoftImage}.